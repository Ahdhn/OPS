\documentclass[11pt]{article}
\usepackage[colorlinks,urlcolor=blue,linkcolor=blue,citecolor=blue]{hyperref}
\usepackage{graphicx}
% \usepackage[footnotesize]{subfigure}
\usepackage{listings}
\usepackage{verbments}

\date{April 2014}

 \topmargin 0.in  \headheight 0pt  \headsep 0pt  \raggedbottom
 \oddsidemargin 0.1in
 \textheight 9.25in  \textwidth 6.00in
 \parskip 5pt plus 1pt minus 1pt
 \def \baselinestretch {1.25}   % one-and-a-half spaced
 \setlength {\unitlength} {0.75in}

\newenvironment{routine}[2]
{\vspace{.0in}{\noindent\bf\hspace{-5pt}  #1}{\\ \noindent #2}
\begin{list}{}{
\renewcommand{\makelabel}[1]{{\tt  ##1 } \hfil}
\itemsep 0pt plus 1pt minus 1pt
\leftmargin  1.5in
\rightmargin 0.0in
\labelwidth  1.1in
\itemindent  0.0in
\listparindent  0.0in
\labelsep    0.05in}
}{\end{list}}
%

\begin{document}

\title{OPS C++ User's Manual}
\author{Mike Giles, Istvan Reguly, Gihan Mudalige}
\maketitle

\newpage


\tableofcontents


\newpage
\section{Introduction}


OPS is a high-level framework with associated libraries and preprocessors to generate parallel executables for
applications on \textbf{multi-block structured grids}. Multi-block structured grids consists of an unstructured
collection of structured meshes/grids. This document describes the current OPS C++ API, which supports the development
of single-block structured mesh applications. Initial proposals for extending to multi-block structured meshes are also
detailed. 

Many of the API and library follows the structure of the OP2 high-level library for unstructured mesh
applications~\cite{op2}. However the structured mesh domain is distinct from the unstructured mesh applications domain
due to the implicit connectivity between neighboring mesh elements (such as vertices, cells) in structured
meshes/grids. The key idea is that operations involve looping over a ``rectangular'' multi-dimensional set of grid
points using one or more ``stencils'' to access data.\\

\noindent To clarify some of the important issues in designing the API, we note here some needs connected with a 3D
application:
\begin{itemize}
\item
When looping over the interior with loop indices $i,j,k$, often 
there are 1D arrays which are referenced using just one of the 
indices.

\item
To implement boundary conditions, we often loop over a 2D face,
accessing both the 3D dataset and data from a 2D dataset.

\item
To implement periodic boundary conditions using dummy ``halo'' 
points, we sometimes have to copy one plane of boundary data to
another.  e.g.~if the first dimension has size $I$ then we might 
copy the plane $i=I\!-\!2$ to plane $i=0$, and plane $i=1$ to 
plane $i=I\!-\!1$.

\item
In multigrid, we are working with two grids with one having twice
as many points as the other in each direction.  To handle this we 
require a stencil with a non-unit stride.

\item
In multi-block grids, we have several structured blocks.  The 
connectivity between the faces of different blocks can be quite 
complex, and in particular they may not be oriented in the same 
way, i.e.~an $i,j$ face of one block may correspond to the $j,k$ 
face of another block.  This is awkward and hard to handle simply.
\end{itemize}

\noindent The latest proposal is to handle all of these different requirements through stencil definitions.


\clearpage

\newpage
\section{OPS C++ API}

\subsection{Initialisation and termination routines}

\begin{routine} {void ops\_init(int argc, char **argv, int diags\_level)}
{This routine must be called before all other OPS routines.}
\item[argc, argv]   the usual command line arguments
\item[diags\_level] an integer which defines the level of debugging
                    diagnostics and reporting to be performed
\end{routine}


\begin{routine} {void ops\_exit()}
{This routine must be called last to cleanly terminate the OPS computation.}
\item \vspace{-0.3in}
\end{routine}

\subsection{Declaration routines}

\begin{routine} {ops\_block ops\_decl\_block(int dims, int *size, char *name)}
{This routine defines a structured grid block.}

\item[dims]          dimension of the set
\item[size]          size of the set in each dimension
\item[name]          a name used for output diagnostics
\end{routine}

\begin{routine} {ops\_dat ops\_decl\_dat(ops\_block block, int dim, int* size, \\
\hspace*{1.6in}int* d\_n, int* d\_p, T *data, char *name)}
{This routine defines a dataset. }

\item[block]         structured block
\item[dim]           dimension of dataset (number of items per block element)
\item[size]	     size of the dat in each dimension
\item[d\_m]	     max offset from negative edge in each dimension
\item[d\_p]	     max offset from positive edge in each dimension
\item[data]          input data of type {\tt T}
\item[name]          a name used for output diagnostics
\end{routine}\vspace{-5pt}
The depths \texttt{d\_m} and \texttt{d\_p} are used to indicate the offset from the edge in both the negative and
positive directions of each dimension. In the current API, it is important that the application developer indicate to
OPS, the maximum stencil depth a given \texttt{ops\_dat} is accessed within any of the subsequent loops over a block.
This is to account for stencils accessing beyond the defined regions of the \texttt{ops\_dat} during an execution,
particularly for distributed memory halo allocation within OPS. In the future OPS will be able to dynamically allocate
the maximum necessary depths based on the stencils declared to be used on a given \texttt{ops\_dat}.\\


\begin{routine} {ops\_decl\_const(char const* name, int dim, char const type, T* data)}{Declare a
global constant}
\item[name]       name of constant
\item[dim]     dimension of the data
\item[type]    data type
\item[data]    constant data of type {\tt T}
\end{routine}

\begin{routine} {void ops\_diagnostic\_output()}
{This routine prints out various useful bits of diagnostic info about sets, mappings and datasets}
\item \vspace{-0.3in}
\end{routine}



\newpage

\subsection{Parallel loop syntax}

A parallel loop with N arguments has the following syntax:

\begin{routine} {void ops\_par\_loop(\ void (*kernel)(...), char *name, \\
\hspace*{1.4in} ops\_block block, int dims, int *range,\\
\hspace*{1.4in} ops\_arg arg1,\ ops\_arg arg2,\ \ldots ,\ ops\_arg argN\ )}{}

\item[kernel]     user's kernel function with N arguments
\item[name]       name of kernel function, used for output diagnostics
\item[block]       ops\_block on which this loop iterates over
\item[dims]       dimension of loop iteration
\item[range]      iteration range array
\item[args]       arguments
\end{routine}


\vspace{0.2in}
\noindent
The {\bf ops\_arg} arguments in {\bf ops\_par\_loop} are provided by one of the 
following routines, one for global constants and reductions, and the other 
for OPS datasets.

\vspace{0.1in}


\begin{routine} {ops\_arg ops\_arg\_gbl(T *data, int dim, char *type, ops\_access acc)}{}
\item[data]       data array
\item[dim]        array dimension
\item[type]    	  string representing the type of data held in data
\item[acc]        access type
\end{routine}


\begin{routine} {ops\_arg ops\_arg\_dat(ops\_dat dat, ops\_stencil stencil, char *type,
                 ops\_access acc)}{}
\item[dat]        dataset
\item[stencil]    stencil for accessing data
\item[type]    	  string representing the type of data held in dataset
\item[acc]        access type
\end{routine}



\newpage

\noindent The final ingredient is the stencil specification, for which we have two versions: simple and strided.\\

\begin{routine} {ops\_stencil ops\_decl\_stencil(int dims,
                 int points, int *stencil, char *name)}{}
\item[dims]     dimension of loop iteration
\item[points]   number of points in the stencil
\item[stencil]  stencil for accessing data
\item[name]	string representing the name of the stencil
\end{routine}


\begin{routine} {ops\_stencil ops\_decl\_strided\_stencil(int dims, int points,\\
\hspace*{2.65in} int *stencil, int *stride, char *name)}{}
\item[dims]       dimension of loop iteration
\item[points]     number of points in the stencil
\item[stencil]    stencil for accessing data
\item[stride]     stride for accessing data
\item[name]	string representing the name of the stencil
\end{routine}

\vspace{0.2in}

\noindent In the strided case, the indices for referencing the data used by point {\tt p} are defined as:\\

\noindent {\tt stride[m]*loop\_index[m] + stencil[p*dims+m]}\\

\noindent If, for one or more dimensions, both {\tt stride[m]} and {\tt stencil[p*dims+m]} are zero, then one of the
following must be true;
\begin{itemize}
\item
the dataset being referenced has size 1 for these dimensions
\item
these dimensions are to be omitted and so the dataset has 
dimension equal to the number of remaining dimensions.
\end{itemize}

\noindent These two stencil definitions probably take care of all of the cases in the Introduction except for multiblock
applications with interfaces with different orientations -- this will need a third, even more general, stencil
specification.

\noindent The strided stencil will handle both multigrid (with a stride of 2) and the boundary condition and reduced
dimension applications (with a stride of 0 for the relevant dimensions).




\newpage
\section{OPS User Kernels}

\noindent In OPS, the elemental operation carried out per mesh/grid point is specified as an outlined function called
a \textit{user kernel}. An example is given in \figurename{ \ref{fig:example}}.
\begin{figure}[h]\small
\vspace{-0pt}\noindent\line(1,0){8}\vspace{-20pt}
\begin{pyglist}[language=c]
void accelerate_kernel_stepbymass( double *density0, double *volume,
                                   double *stepbymass) {
  double nodal_mass;

  //uses the four point stencil {0,0, -1,0, 0,-1, -1,-1};
  nodal_mass = ( density0[OPS_ACC0(-1,-1)] * volume[OPS_ACC1(-1,-1)]
    + density0[OPS_ACC0(0,-1)] * volume[OPS_ACC1(0,-1)]
    + density0[OPS_ACC0(0,0)] * volume[OPS_ACC1(0,0)]
    + density0[OPS_ACC0(-1,0)] * volume[OPS_ACC1(-1,0)] ) * 0.25;

  stepbymass[OPS_ACC2(0,0)] = 0.5*dt / nodal_mass;
}
\end{pyglist}
\vspace{-10pt}\noindent\line(1,0){8}\vspace{-10pt}
\caption{\small example user kernel}
\normalsize\vspace{-0pt}\label{fig:example}
\end{figure}
\begin{figure}[h]\small
\vspace{-0pt}\noindent\line(1,0){8}\vspace{-20pt}
\begin{pyglist}[language=c]
int rangexy_inner_plus1[] = {x_min,x_max+1,y_min,y_max+1}; //x-y range

ops_par_loop(accelerate_kernel_stepbymass, "accelerate_kernel_stepbymass", 
             colver_grid, 2, rangexy_inner_plus1,
             ops_arg_dat(density0, S2D_00_M10_0M1_M1M1, "double", OPS_READ),
             ops_arg_dat(volume,   S2D_00_M10_0M1_M1M1, "double", OPS_READ),
             ops_arg_dat(work_array1, S2D_00, "double", OPS_WRITE));
\end{pyglist}
\vspace{-10pt}\noindent\line(1,0){8}\vspace{-10pt}
\caption{\small example \texttt{ops\_par\_loop}}
\normalsize\vspace{-0pt}\label{fig:parloop}
\end{figure}

\noindent This user kernel is then used in an \texttt{ops\_par\_loop} (\figurename{ \ref{fig:parloop}}). The key aspect
to note in the user kernel in \figurename{ \ref{fig:example}} is the use of the macros \texttt{OPS\_ACC0, OPS\_ACC1} and
\texttt{OPS\_ACC2}. These specifies the stencil in accessing the elements of the respective data arrays. At compile
time these macros will be expanded to give the correct array index (in this case accessing a 1D array with 2D indexing
related to the stencil specified) to access the relevant element.

\begin{thebibliography}{1}
\bibitem{op2} OP2 for Many-Core Platforms, 2013. \url{http://www.oerc.ox.ac.uk/projects/op2}
\end{thebibliography}

\end{document}




